
\documentclass[10pt,fleqn]{article}

\usepackage[russian]{babel}
\usepackage[utf8]{inputenc}
\usepackage{color}
\usepackage{amsmath}
\usepackage{amssymb}
\usepackage{graphics}
\usepackage{epsfig}
\usepackage{bm}
\usepackage[colorlinks,urlcolor=blue]{hyperref}
\usepackage{tikz}
\usepackage{pgfplots}
\usepackage{verbatim}
\usepackage{mdframed}
\usepackage{dirtree}
\usepackage{indentfirst}
\usepackage{url}

\definecolor{codegray}{gray}{0.9}
\newcommand{\code}[1]{%
  \begingroup\setlength{\fboxsep}{1pt}%
  \colorbox{codegray}{\texttt{\hspace*{2pt}\vphantom{Ay}#1\hspace*{2pt}}}%
  \endgroup
}

% mdinlinecode command for including code snippets inline
% (fake verbatim, so all special character should be escaped,
% or textmode equivalents of special characters should be used)
\definecolor{mdinlinecodeboxframecolor}{HTML}{DDDDDD}
\definecolor{mdinlinecodeboxbackgroundcolor}{HTML}{F8F8F8}
\newcommand{\mdinlinecode}[1]{%
    \begin{tikzpicture}[baseline=0ex]%
        \node[anchor=base,%
            text height=0.9em,%
            text depth=0.9ex,%
            inner ysep=0pt,%
            draw=mdinlinecodeboxframecolor,%
            fill=mdinlinecodeboxbackgroundcolor,%
            rounded corners=1.5pt] at (0,0) {\small\texttt{#1}};%
    \end{tikzpicture}%
}

\newmdenv[font=\footnotesize,%
linewidth=0.4pt,%
roundcorner=2pt,%
linecolor=mdinlinecodeboxframecolor,%
backgroundcolor=mdinlinecodeboxbackgroundcolor,%
settings={\setlength{\parindent}{0pt}}]{mdcdblk}
\newenvironment{mdcodeblock}{\endgraf\verbatim}{\endverbatim}
\BeforeBeginEnvironment{mdcodeblock}{\begin{mdcdblk}}
\AfterEndEnvironment{mdcodeblock}{\end{mdcdblk}}

\textheight=26cm % высота текста
\textwidth=18cm % ширина текста
\oddsidemargin=-1cm % отступ от левого края
\topmargin=-3cm % отступ от верхнего края
\sloppy

\newcounter{example}

%-- Обозначение вектора жирным символом
\def\vec#1{\mathchoice{\mbox{\boldmath$\displaystyle#1$}}
{\mbox{\boldmath$\textstyle#1$}} {\mbox{\boldmath$\scriptstyle#1$}} {\mbox{\boldmath$\scriptscriptstyle#1$}}}

\DeclareMathOperator{\B}{Bin}
\DeclareMathOperator{\Ps}{Poiss}
\DeclareMathOperator{\R}{Unif}
\DeclareMathOperator{\sign}{\mathrm{sign}}
\DeclareMathOperator{\softmax}{\mathrm{softmax}}
\DeclareMathOperator{\loss}{\mathcal{L}}

\pagestyle{empty}

\begin{document}

\begin{center}
    \begin{tabular}{|p{17.5cm}|}
        \hline
        \textbf{ВМК}\\
        \begin{center} \Large Задание 3. Ансамбли алгоритмов. Веб-сервер. \\ Композиции алгоритмов для решения задачи регрессии. \end{center}\\
        \textbf{Практикум 317 группы, 2022}\\
        \hline
    \end{tabular}
\end{center}

\

\begin{tabbing}
    Начало выполнения задания: 22 ноября 2022 года, 22:00.\\
    Мягкий Дедлайн: \textcolor{blue}{\bf 16 декабря 2022 года, 08:00.} \\
    Жёсткий Дедлайн: \textcolor{red}{\bf 23 декабря 2022 года, 08:00.}
\end{tabbing}

\section*{Формулировка задания}
Данное задание направлено на ознакомление с алгоритмами композиций.

В задании необходимо:
\begin{enumerate}
 \item Написать на языке \mdinlinecode{Python} собственную реализацию методов \textbf{случайный лес} и \textbf{градиентный бустинг}. Прототипы функций должны строго соответствовать прототипам, описанным в спецификации. При написании необходимо пользоваться стандартными средствами языка \mdinlinecode{Python}, библиотеками \mdinlinecode{numpy}, \mdinlinecode{scipy} и \mdinlinecode{matplotlib}. Библиотекой \mdinlinecode{scikit-learn} пользоваться запрещается, если это не обговорено отдельно в пункте задания.
 \item Провести \hyperref[sec:exps]{описанные ниже эксперименты} с выданными данными. Написать отчёт о проделанной работе (формат PDF). Отчёт должен быть подготовлен в системе \LaTeX.
 \item Написать реализацию веб-сервера с \hyperref[subsec:webreq]{требуемой функциональностью}. Обернуть своё решение в docker контейнер.
 \item Весь код, написанный во время выполнения задания, должен быть размещён в приватном репозитории. Требования к ведению репозитория также \hyperref[subsec:projcond]{описаны ниже}.
\end{enumerate}

\section*{Экспериментальная часть}
Эксперименты для этого задания необходимо проводить на датасете данных о продажах недвижимости \textbf{House Sales in King County, USA}. Данные можно скачать по \href{https://www.kaggle.com/harlfoxem/housesalesprediction}{ссылке}.

\subsection*{Реализация алгоритмов (10 баллов)}
Прототипы всех необходимых функций описаны в файлах, прилагающихся к заданию. Среди предоставленных файлов должны быть следующие модули и функции в них:

\begin{enumerate}
    \item Модуль \mdinlinecode{ensembles.py} с реализациями случайного леса и градиентного бустинга. Алгоритмы должны соответствовать классическим реализациям, разобранным на лекциях ММРО \cite{VoronLinearEns}, \cite{VoronAdvanceEns}.
\end{enumerate}

\textbf{Замечание:} Для одномерной оптимизации используйте функцию \mdinlinecode{minimize\_scalar}. Разрешается использовать класс \mdinlinecode{DecisionTreeRegressor} из библиотеки \mdinlinecode{scikit-learn}.

\subsection*{Эксперименты (15 баллов)\label{sec:exps}}

\begin{enumerate}
    \item Проведите предобработку имеющихся данных. Разделите данные на обучение и контроль, переведите данные в \mdinlinecode{numpy.ndarray}. Опишите выполненную предобработку данных в отчёте.
    \item Исследуйте поведение алгоритма \textbf{случайный лес}. Изучите зависимость \textbf{RMSE} на отложенной выборке и \textbf{время работы алгоритма} в зависимости от следующих факторов:
    \begin{itemize}
        \item количество деревьев в ансамбле
        \item размерность подвыборки признаков для одного дерева
        \item максимальная глубина дерева (дополнительно разберите случай, когда глубина неограничена)
    \end{itemize}
    \item Исследуйте поведение алгоритма \textbf{градиентный бустинг}. Изучите зависимость \textbf{RMSE} на отложенной выборке и \textbf{время работы алгоритма} в зависимости от следующих факторов:
    \begin{itemize}
        \item количество деревьев в ансамбле
        \item размерность подвыборки признаков для одного дерева
        \item максимальная глубина дерева (дополнительно разберите случай, когда глубина неограничена)
        \item выбранный \mdinlinecode{learning\_rate} (каждый новый алгоритм добавляется в композицию с коэффициентом $\alpha \cdot \mdinlinecode{learning\_rate}$)
    \end{itemize}
\end{enumerate}

\textbf{Замечание:} Для исследования зависимости от количества деревьев не обязательно с нуля переобучать модель.


\section*{Инфраструктурная часть}
\subsection*{Реализация веб-сервера (15 баллов)\label{subsec:webreq}}
В этой части задании вам предлагается спроектировать веб-интерфейс для взаимодействия с вашей моделью. Считайте, что назначение вашего интерфейса --- обучение моделей человеком, который не знает языка \mdinlinecode{Python}. Это творческое задание, вы можете использовать при реализации всё, что считаете нужным.

В интерфейсе обязательно должна быть возможность выполнять следующие действия:
\begin{enumerate}
    \item (7 баллов) Создание новой модели. Необходимо предусмотреть возможность указывать тип (случайный лес или градиентный бустинг) и гиперпараметры модели. В интерфейсе должна быть возможность обучения модели на произвольном датасете, совпадающем по формату с датасетом из условия (то есть \mdinlinecode{.csv} файл в котором один из столбцов задаёт целевую переменную, а подмножество остальных столбцов задает признаки объектов выборки). Также, реализуйте возможность передавать данные для валидации.
    \item (4 балла) Просмотр информации о модели. Пользователь должен иметь возможность получать информацию о гиперпараметрах модели, датасете, на котором она обучалась, а также о значении функции потерь на обучающей и валидационной (при наличии) выборках после каждой итерации.
    \item (4 балла) Выполнение предсказаний с использованием ранее обученной модели. Считайте, что данные для предсказания представлены в том же формате, что и данные для обучения модели (за исключением отсутствия столбца с целевой переменной).
\end{enumerate}

\textbf{Замечание:} При реализации интерфейса необходимо предусмотреть все возможные проверки на входные данные и действия пользователя. Например, сайт не должен падать при передаче невалидного файла с обучающими данными.

\subsection*{Ведение проекта (10 баллов)\label{subsec:projcond}}
Весь код вашего решения должен быть выложен в приватный \mdinlinecode{github} репозиторий. Ваш проект должен быть организован в соответствии с \hyperref[fig:dirs]{Рисунком \ref{fig:dirs}}. По необходимости вы можете создавать другие дополнительные файлы и директории. Полный балл за данный пункт может быть выставлен только при условии качественного ведения репозитория. Качественное ведение включает в себя следующие требования:
\begin{enumerate}
    \item Основная разработка ведётся не в \mdinlinecode{master}, а в отдельных ветках. Ветка соответствует решению одной глобальной задачи.
    \item Одно важное изменение в коде --- один коммит в системе.
    \item Обновление \mdinlinecode{master} ветки происходит посредством \mdinlinecode{pull request} и \mdinlinecode{merge}.
    \item Сообщения коммитов и описание \mdinlinecode{pull request} написаны понятно и содержательно.
\end{enumerate}

Решение должен быть обёрнуто в docker контейнер. В репозитории должен содержаться \mdinlinecode{Dockerfile}, а также инструкция по его сборке. Образ вашего контейнера должен быть загружен на \mdinlinecode{dockerhub.com}.

Качество кода влияет на итоговую оценку, код должен быть структурированным и понятным. Ваш код должен удовлетворять кодстайлу. В частности, проходить проверку линтерами:

\begin{mdcodeblock}
    # Linter for Dockerfile
    cat Dockerfile | docker run --rm -i hadolint/hadolint
    # Linter for shell scripts
    docker run --rm -v "/path/to/script/folder:/mnt" koalaman/shellcheck:stable script.sh
    # Linter for Python scripts
    flake8 script.py --max-line-length=120 ; pylint script.py --max-line-length=120 --disable="C0103,C0114,C0115"
\end{mdcodeblock}

В репозитории должен быть указан \mdinlinecode{README.md} файл, объясняющий как необходимо пользоваться вашей системой. Подробнее про Markdown разметку можно прочитать в \href{https://www.markdownguide.org/basic-syntax}{документации}. В \mdinlinecode{README.md} необходимо подробно описать не только процесс сборки и запуска контейнера, но и инструкцию по использованию всех реализованных функций в приложении. Использование иллюстраций и скриншотов в инструкции крайне желательно.

\begin{figure}[h!]
    \begin{center}
        \begin{minipage}[b]{0.8\textwidth}
            \renewcommand*\DTstylecomment{\color{blue}}
            \renewcommand*\DTstyle{\ttfamily\textcolor{red}}
            \dirtree{%
            .1 github.com/<YourNickname>/<RepoName>.
            .2 src\DTcomment{Исходники веб-приложения и реализация алгоритмов}.
            .3 ensembles.py.
            .3 ....
            .2 scripts\DTcomment{Скрипты для сборки и запуска docker контейнера}.
            .3 build.sh.
            .3 run.sh.
            .3 ....
            .2 Dockerfile\DTcomment{Описание контейнера с веб-приложением}.
            .2 report.pdf\DTcomment{Отчёт по заданию}.
            .2 ....
            .2 README.md\DTcomment{Описание проекта. Документация}.
            }
        \end{minipage}
    \end{center}
    \caption{\label{fig:dirs}Требуемая структура github репозитория}
\end{figure}

\subsection*{Бонусная часть (до 10 баллов)}
В этом задании нет чётких условий для бонусной части. Дополнительные баллы могут быть поставлены за любой хорошо реализованный дополнительный функционал, неописанный в задании. Не забудьте в отчёте и в посылке в anytask написать, что за дополнительный функционал вы реализовали.

Несколько идей для получения дополнительных баллов:
\begin{enumerate}
    \item (до 4 баллов) Использование интерактивных графиков, например, с помощью библиотеки \mdinlinecode{plotly}.
    \item (до 4 баллов) Реализация веб-интерфейса без перезагрузки страниц, например, с помощью \mdinlinecode{AJAX}.
\end{enumerate}


\addcontentsline{toc}{section}{\bibname}

\nocite{VoronLinearEns, VoronAdvanceEns}

\def\BibUrl#1.{}\def\BibAnnote#1.{}
\bibliographystyle{gost71s}
\bibliography{bibliography}

\end{document}
