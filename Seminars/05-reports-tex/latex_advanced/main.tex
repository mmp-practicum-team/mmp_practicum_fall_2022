\documentclass[aspectratio=169]{beamer}

\usepackage{mystyle}
\usepackage{commands}
\usepackage{custombeamer}

\title[Введение в Python]{\bfseries Занятие 5.3: Некоторые детали \LaTeX}
\author{
    Автор презентации: Васильев Руслан Леонидович\\
    Преподаватель: Находнов Максим Сергеевич
}
\subtitle{Практикум на ЭВМ, осень 2022}
\institute[OM]{кафедра ММП, ВМК МГУ}
\date{06 октября 2022 г.}

\begin{document}

\frame{\titlepage}

\begin{frame}
    \frametitle{Содержание}
    \tableofcontents
\end{frame}

\section{Основы}
\begin{frame}[fragile]{Minimal example}
\begin{lcode}
    \documentclass{article}

    \begin{document}
        Hello, disjunctive normal forms!
    \end{document}
\end{lcode}

\textcolor{red}{Будет ли корректно рендериться русский текст в таком примере?}

\end{frame}

\begin{frame}[fragile]{Minimal example: русский язык}
    Чтобы русский язык можно было отображать, нужно указать кодировку текстовых шрифтов для выходных символов:
\begin{lcode}
    \documentclass{article}

    \usepackage[T2A]{fontenc}

    \begin{document}
        Сокращенная ДНФ и способы ее построения
    \end{document}
\end{lcode}
\end{frame}

\begin{frame}[fragile]{Medium example}
    \begin{lcode}
    \documentclass{article}

    % Кодировка выходных символов
    \usepackage[T1, T2A]{fontenc}
    % Кодировка входного файла
    \usepackage[utf8]{inputenc}
    % Правила переноса слов, названия секций и другая локализация
    \usepackage[english, russian]{babel}

    \begin{document}
        Использование градиентного алгоритма для построения ДНФ.
    \end{document}
    \end{lcode}
\end{frame}

\begin{frame}[fragile]
    \begin{exampleblock}{Замечение}
    По умолчанию \texttt{babel} и \texttt{fontenc} используют последние аргументы, предыдущие же позволяют переключаться внутри документа на другие языки/кодировки.
    \end{exampleblock}
    \vfill{}
    \begin{itemize}
        \item \verb|\fontencoding{T1} \selectfont|
        \begin{itemize}
            \item T2A: \quad {\fontencoding{T2A}\selectfont ?`}
            \item T1: \quad {\fontencoding{T1}\selectfont ?`}
        \end{itemize}
        \item \verb|\selectlanguage{english}|
    \end{itemize}
\end{frame}

\begin{frame}[fragile]\frametitle{Группы и окружения}
\begin{lcode}
    \begin{имя_окружения}
        ...
    \end{имя_окружения}
\end{lcode}
\begin{exampleblock}{Свойство окружений}
    Часть файла, находящаяся внутри окружения, образует группу.
\end{exampleblock}

Примеры: \verb|{document}|, \verb|{figure}|, \verb|{tabular}|, \verb|{minted}|, \verb|{minipage}|, \verb|{equation}|, \verb|{itemize}|, \verb|{frame}|, \verb|{center}|, \verb|{theorem}|, \dots
\end{frame}

\begin{frame}[fragile]\frametitle{Группы и окружения}
    \begin{minipage}{0.45\linewidth}
        Сначала {переключим шрифт
        на \itshape курсив; теперь
        сделаем шрифт еще и
        {\bfseries полужирным;}
        посмотрите, как восстановится}
        шрифт после кон{ца г}руппы.
    \end{minipage}
    \hfill
    \begin{minipage}{0.45\linewidth}
\begin{lcode}
Сначала {переключим шрифт
на \itshape курсив; теперь
сделаем шрифт еще и
{\bfseries полужирным;}
посмотрите, как восстановится}
шрифт после кон{ца г}руппы.
\end{lcode}
    \end{minipage}
\end{frame}


\section{Разбиение проекта}
\begin{frame}[fragile]
    \begin{exampleblock}{Основная команда}
        \verb|\input{имя_файла}|
    \end{exampleblock}
\begin{lcode}
    \documentclass[12pt]{report}

    \usepackage{mystyle}
    \usepackage{commands}

    \begin{document}
        \input{titlepage}
        \input{intro}
        \input{theory}
        \input{experiments}
        \input{conclusion}
    \end{document}
\end{lcode}
\end{frame}

\begin{frame}[fragile]\frametitle{Другие варианты}
    \begin{itemize}
        \item Пакет \verb|subfiles| позволяет компилировать файлы отдельно, причем все будут иметь общую преамбулу основного файла.
        \item Пакет \verb|standalone|, напротив, объединяет преамбулы.
        \item \verb|\include{file}| допускает рекурсивные подстановки и позволяет использовать \verb|\includeonly{file1, file2}|
    \end{itemize}
\end{frame}

\section{Команды}

\begin{frame}[fragile]\frametitle{newcommand}
\begin{lcode}
    \newcommand{cmd}{def}
    \newcommand{cmd}[args]{def}
    \newcommand{cmd}[args][default]{def}
\end{lcode}
\medskip
Переопределение существующей команды делается аналогично с помощью \verb|\renwecommand|.
\end{frame}


\begin{frame}[fragile]\frametitle{newcommand}
\begin{lcode}
    \newcommand{\mpl}[2]{
        \begin{figure}[!h]
            \includegraphics[width=\textwidth]{#1}
            \centering
            \caption{#2}
            \label{fig:#1}
        \end{figure}
    }
\end{lcode}

\textcolor{red}{А зачем здесь используется окружение figure?}

\end{frame}

\begin{frame}[fragile]\frametitle{DeclareMathOperator}
    \begin{alertblock}{Что бы мы делали без \texttt{amsmath}}
        \centering
        \verb|\newcommand{\softmax}{\mathop{\mathrm{softmax}}\nolimits}|
    \end{alertblock}
    \bigskip
    \begin{exampleblock}{Жизнь удалась}
        \centering
        \verb|\DeclareMathOperator{\softmax}{Softmax}|
    \end{exampleblock}
    \begin{equation*}
        \softmax(\vec{s})=\left(\frac{e^{s_1}}{\sum_{t=1}^{m}e^{s_t}},\dots,\frac{e^{s_m}}{\sum_{t=1}^{m}e^{s_t}}\right)
    \end{equation*}
\end{frame}


\section{Пустое пространство}

\begin{frame}[fragile]\frametitle{Разрывы}
    Существует множество команд для разрыва строк:

    \begin{itemize}
        \item \verb|\\| (или \verb|\\*|)~--- то, что нужно
        \item \verb|\newline|~--- тоже, что и \verb|\\|
        \item \verb|\hfill \break|~--- заполнение строки и разрыв
        \item \verb|\linebreak[number]|~--- разрыв с приоритетом (от $0$ до 4)
    \end{itemize}
    \vspace{10pt}
    А также разрыва страниц:
    \begin{itemize}
        \item \verb|\newpage|~--- то, что нужно
        \item \verb|\clearpage|~--- перед разрывом принудительно выводит все плавающие окружения
        \item \verb|\pagebreak[number]|~--- разрыв с приоритетом
    \end{itemize}
\end{frame}

\begin{frame}[fragile]
    \begin{minipage}{0.45\linewidth}
Эта строка\\ была разорвана.
Справа осталось пустое
место, но зато строка
не разреженная.

\vspace{15pt}

Эта строка была \linebreak разорвана. Она выровнена по
правому краю, но для этого ее
пришлось безбожно растянуть.
\end{minipage}\hfill
\begin{minipage}{0.45\linewidth}
\begin{lcode}
Эта строка\\ была разорвана.
Справа осталось пустое
место, но зато строка
не разреженная.
\end{lcode}

\vspace{5pt}

\begin{lcode}
Эта строка была\linebreak
разорвана. Она выровнена по
правому краю, но для этого ее
пришлось безбожно растянуть.
\end{lcode}
\end{minipage}
\end{frame}

\begin{frame}[fragile]\frametitle{Дополнительные пробелы}
    Горизонтальные:
    \begin{itemize}
        \item \verb|\hspace{1cm}|~--- фиксированная длина
        \item \verb|\hfill|~--- до конца
    \end{itemize}
    \medskip
    Вертикальные:
    \begin{itemize}
        \item \verb|\vspace{10pt}|~--- фиксированная длина
        \item \verb|\vfill|~--- до конца
        \item \verb|\smallskip|, \verb|\medskip|, \verb|\bigskip|
    \end{itemize}
    \begin{block}{Математические}
    Горизонтальные:
    \verb|\! \, \: \; \ \quad \qquad|. \\
    Вертикальные удобнее всего задавать с помощью \verb|\\[10pt]|.
    \end{block}
\end{frame}


\section{Таблицы}
\begin{frame}
    \newcommand{\xxx}{
    \begin{tabular}{l}
        $X_1=10$ \\
        $X_2=20$ \\
        $X_3=30$
    \end{tabular}
}

\begin{table}[h!]
    \centering
    \begin{tabular*}{0.5\textwidth}{@{\extracolsep{\fill} }lcccc}
        \toprule 
         & \multicolumn{2}{c}{$X_1=100$} & \multicolumn{2}{c}{\xxx} \\ \cmidrule(lr){2-3} \cmidrule(lr){4-5}
         & $a=0.5$ & $a=1$  & $a=0.5$ & $a=1$ \\ \midrule
        \textbf{A} & 1 & 2 & 3& 4 \\ 
        \textbf{B} & 5 & 6 & 7 & 8  \\  
        \textbf{C} & 9 & 10 & 11 & 12 \\ \bottomrule
    \end{tabular*}%}
    \caption{Подпись к таблице}
    \label{tab:vals}
\end{table}

\end{frame}

\section{Презентации}
\begin{frame}\frametitle{beamer}
    \begin{itemize}
        \item Наиболее мощный пакет для создания презентаций;
        \item Верстка почти не отличается от обычного \TeX'а;
        \item Основное окружение~--- \texttt{frame} (слайд ли?) \pause;
        \item Огромное число предустановленных тем;
        \item Готовые шаблоны и блоки (например, block, examples, alert).
    \end{itemize}
\end{frame}


\begin{frame}
    \begin{block}{Блок, соответствующий теме}
    Информация!
    \end{block}
    \vfill
    \begin{alertblock}{Важное замечание}
    На что-то стоит обратить внимание.
    \end{alertblock}
    \vfill
    \begin{exampleblock}{Блок с примерами}
    Раз, два, три.
    \end{exampleblock}
\end{frame}

\section{Подсветка кода}
\begin{frame}[fragile]\frametitle{Основные пакеты}
    \begin{itemize}
        \item \verb|verbatim|~--- запрещает обрабатывать вставленный текст;
        \item \verb|lstlisting|~--- подсветка, сложная настройка;
        \item \verb|minted|~--- подсветка, простая настройка, сложная установка.
    \end{itemize}
\end{frame}

\begin{frame}[fragile]
\begin{lcode}
    \mint{html}|<h2>Текст для <b>онлайн-разметки</b>.</h2>|
\end{lcode}

\mint{html}|<h2>Текст для <b>онлайн-разметки</b>.</h2>|
\medskip
\begin{lcode}
\begin{minted}{python}
import numpy as np

def mytranspose(matrix):
    return np.swapaxes(matrix, 0, 1)
\end{minted}
\end{lcode}

\begin{minted}{python}
import numpy as np

def mytranspose(matrix):
    return np.swapaxes(matrix, 0, 1)
\end{minted}

\end{frame}

\section{Список литературы}
\begin{frame}[fragile]
    \begin{enumerate}
        \item Подключаем \verb|\usepackage[bibstyle=gost-numeric]{biblatex}|
        \item Создаем \verb|papers.bib|
        \item Указываем путь \verb|\addbibresource{papers.bib}|
        \item Цитируем \verb|\cite{paper}|
        \item Выводим \verb|\printbibliography|
    \end{enumerate}

\end{frame}

\end{document}
